\label{chap:related_work}
The memory partitioning method is widely used for the memory power
optimization problem. Many researchers have obtained good results
and benefits through the usage of this approach
\cite{Strobel2016, 876761, 4415607}.
The detail development of the mathematic power model discussed in
Section \ref{sec:memory_partition} is illustrated in \cite{Strobel2016}.
In this article, M. Strobel et al. propose the the ILP approach to
be used in the optimization of the memory partitioning. And the results
from their experiments show that the usage of ILP can yield an optimal
configuration for their problem set. Since the memory partitioning is
one kind of the combinatorial optimization, heuristics can also be a
potential approach for it.

The combinatorial optimization is one of the hottest topics which
aims to search for an optimal solution in a finite solution space.
Most pf the conventional methods such as exhaustive search is not a proper
approach for it.
S. Kirkpatrick et al. find that there existing a close relationship
between the statistical mechanics and the combinatorial optimization
\cite{10.2307/1690046}.
Based on this finding, they propose the simulated annealing algorithm
as a promising approach for the combinatorial optimization in
\cite{10.2307/1690046}.
They introduce the metropolis criterion and explain how it can be
applied to improve the solution quality provided by local search
algorithm. In the article, the simulated annealing process is
developed to be consisted of a parameterized framework which is already
discussed in Section \ref{subsec:simulated_annealing}. To illustrate
the usability of the simulated annealing algorithm, S. Kirkpatrick et al.
deploy the algorithm in the physical design of computers to optimize the
circuits partitioning, placement and wiring processes.
In addition, they also apply the simulated annealing algorithm to the classical
traveling salesmen problem. Multiple experiments for above optimization
problems are conducted in this article and the experiment results are
used to show that good solutions can be obtained by the usage of the
simulated annealing algorithm. It is also pointed out by S. Kirkpatrick et al.
that the accuracy and efficiency of the algorithm are much dependent on
its parameters. Their suggestions for the algorithm setting are discussed
together with other's work later in this chapter.

In \cite{doi:10.1287/opre.37.6.865}, the author implement the simulated 
annealing algorithm for the graph partitioning problem.
And a deep evaluation for the algorithm is made through a compact series
of experiments in the article.
They propose an alternative for the design of the cooling schedule and 
compared it with the original method using in \cite{10.2307/1690046}.
The cooling schedule implemented by S. Kirkpatrick et al. is to reduce
the temperature linearly by a colling ration.
In \cite{doi:10.1287/opre.37.6.865}, the author develop an adaptive
cooling schedule which slows down the temperature reduction when the solution
cost is changing fast in the current searching region. However, no improvement
to the linear cooling schedule is found in their conducted experiments.
For the terminations of the nested loops, they terminate the inner loop when
the maximum number of iterations are achieved. And a low threshold of the
acceptance probability is used for the termination of the outer loop.
Unfortunately, there is no method proposed to the determination of the initial
temperature in \cite{doi:10.1287/opre.37.6.865}.

In \cite{10.2307/1690046}, S. Kirkpatrick et al. suggest two methods to determine
the initial temperation $T_{0}$. The first one is to use the maximum difference
between two neighboring solution costs as the initial temperature. The other one
is described as following. An initial acceptance probability $P_{0}$ is defined
which is the expected acceptance probability at $T_{0}$.
Its value should be a real number close but less than 1, typically in the range of
0.8 to 0.95.
Then a random guess of $T_{0}$ is made and the inner loop of
simulated annealing is performed at this temperature. The solution
acceptance ration of the inner loop performance is measured. And if the
acceptance ration is lower than $P_{0}$, the value of $T_{0}$ is doubled.
The same process is repeated until the measured acceptance
ration is higher than $P_{0}$.
Based on the second suggest of S. Kirkpatrick et al., Johnson et al.
provide Equation \ref{equa:t_0_average} in \cite{Johnson:1991:OSA:108188.108193}
for the estimation of the initial temperature.
\begin{equation}
\label{equa:t_0_average}
	T_{0}= - \dfrac{\overline{\Delta E}}{\ln P_{0}}
\end{equation}
The $\overline{\Delta E}$ in the equation is the average difference 
between two neighboring solution costs. The value of
$\overline{\Delta E}$ can be measured by generating a set of positive
solution acceptances.
Another determination is proposed
by the authors of \cite{white1984concepts}.
They estimate the initial temperature $T_{0}$ by using Equation
\ref{equa:t_0_secod_moment}, where
\textquotedblleft
$K$ is a fixed number in the range of 5 to 10 and
$\sigma^{2}_{\infty}$ is the second moment of the cost distribution 
when the temperature is infinite
\textquotedblright \cite{Ben-Ameur2004}.
However, this approach requires the analysis base on the pre-knowledge
of the solution cost distribution.
\begin{equation}
\label{equa:t_0_secod_moment}
T_{0} = K \sigma^{2}_{\infty}
\end{equation}