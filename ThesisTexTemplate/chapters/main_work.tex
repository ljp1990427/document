\label{chap:main_work}
In this chapter, the main contents of the thesis work are represented.
Figure \ref{fig:framework} shows the process of the memory power
optimization using the simulated annealing algorithm.
There are two kinds of input for the simulated annealing algorithm.
One is the input data related to the optimization problem.
These raw data is recorded in different text files. Their data
structure is not suitable for the simulate annealing algorithm.
Thus, a proper input data organization is defined and a parsing
method is used to transform the raw data into this data organization.
Section \ref{sec:input_organ} discusses the input data organization
and the parsing method in detail.
The other kind of input for the algorithm is the algorithm parameter
that is discussed together with the algorithm design.
Section \ref{sec:sa_design_flow} introduces an abstract design flow of
the simulated annealing.
Section \ref{sec:stage_1} to Section \ref{sec:stage_3} discuss the
design details and problems of three different approaches for the memory
power optimization.
\begin{figure}[H]
	\begin{center}
		\includegraphics[width=0.7\textwidth]{optim_framework}
		\caption{Process of The Simulated Annealing for Memory Power Optimization}
		\label{fig:framework}
	\end{center}
\end{figure}
	\section{Input data organization}
	\label{sec:input_organ}
	As discussed in Section \ref{sec:memory_partition}, the formal
	power model requires the parameters that are relevant to the
	memory types, the application profiles and the interconnect.
	Thus, these parameters are the input data to the simulated
	annealing algorithm.
	Since the object oriented programming is planned to be used for
	the implementation of the simulated annealing, the parameters
	that are related to the same type can be grouped together into
	one class. For example, the physical parameters that are
	relevant to memory types can reside in the memory class.
	Figure \ref{fig:uml} shows the input parameters
	organization in the form of the UML class diagram.
	\begin{figure}[htb]
		\begin{center}
			\includegraphics[width=0.7\textwidth]{uml_class}
			\caption{Input Parameter Organization in UML Diagram}
			\label{fig:uml}
		\end{center}
	\end{figure}
	
	The memory class includes all the physical parameters that are used in the
	formal power model.  Since there are a set of memory types in the
	data file, a set of objects of the memory class will be created in
	the simulated annealing algorithm as well. Thus the memory container
	class is defined to store memory objects.
	The algorithm can also retrieve the required objects and the
	total number of memory objects through the corresponding operations
	in the memory container class.
	The profile class and the application class are defined similarly
	to the memory class and the memory container class respectively.
	The interconnect class is different. It uses two separated lists to store the
	current and area parameters whose value are dependent on the total
	number of instances. These parameters can also be retrieved by the
	simulated annealing algorithm through the corresponding operations
	defined in the interconnect class.
	\begin{figure}[h]
		\begin{center}
			\subfloat[][]
			{
				\includegraphics[width=0.7\textwidth]{mem_data}
				\label{subfig:mem_data}
			}
			\qquad
			\subfloat[][]
			{
				\includegraphics[width=0.7\textwidth]{profile_data}
				\label{subfig:profile_data}
			}
			\qquad
			\subfloat[][]
			{
				\includegraphics[width=0.4\textwidth]{ic_data}
				\label{subfig:ic_data}
			}
		\end{center}
		\caption{Fragments of The Input Parameter Data Files \cite{Strobel2016}}
		\label{fig;input_data}
	\end{figure}

	In this thesis work, the same data sets provided by the authors of \cite{Strobel2016}
	are used as the input parameter data of the simulated
	annealing algorithm. They are recorded in multiple
	plain text data files. Figure \ref{fig;input_data} represents some fragments
	of these data files. Figure \ref{subfig:mem_data} is the fragment of the
	memory data file. Figure \ref{subfig:profile_data} shows the fragment of the
	application profile data file and Figure \ref{subfig:ic_data} is the
	fragment of the interconnect data file. In order to group this data
	into the designed input data organization, a
	parsing method is used. The basic idea of the parsing method is to read the
	plain text file line by line. The required data is extracted and the
	unnecessary data is discarded. Since the data file is text-based,
	the required data are converted into the corresponding data type during the
	extraction.

	\setlength{\textfloatsep}{0.2cm}
\begin{algorithm2e}[H]
	\KwIn{a memory\textbackslash profile data file}
	\KwOut{a memory container\textbackslash application object}
		create a memory container\textbackslash application object\;
		\While{not end of the file}
		{
			read one line\;
			\If{relevant data is contaied in the line}
			{
				extract the data\;
				convert the extracted data into the corresponding type\;
				create a memory\textbackslash profile object based on the converted data\;
				add the created object into the list of the memroy container\textbackslash application object\;
			}
		}
		\Return the memory container\textbackslash application object\;
	\caption{Parse Memory\textbackslash Profile Data File}
	\label{algo:mem_profile_parser}
\end{algorithm2e}
\setlength{\textfloatsep}{0.2cm}
	
	Algorithm \ref{algo:mem_profile_parser} is the pseudo-code of
	the parsing method for the memory and profile data files. If the method is
	used to parse the memory data file, it first reads one line of the file text.
	Then it checks whether there is the relevant data contained in the line or not.
	If there is no data, the the algorithm continues reading the next line.
	Otherwise, the relevant data is extracted and converted into the corresponding
	data type.
	It can be seen from the file fragment in Figure \ref{subfig:mem_data}, all the
	parameters data related to one memory type are listed in one single line.
	Thus, a memory object can be created based on the parsed data
	for one line. Then the created object is added to the list in the memory container
	object. The same parsing step is repeated until the method reaches the end of
	the file. The parsing process of the application profile data file is similar
	to the parsing of the memory data file.
	The only differences are that it deals with the profile data file and
	profile objects are created and added to the list of the application object.
	However, the parsing method of the interconnect data file is modified slightly.
	Algorithm \ref{algo:ic_parser} shows the pseudo-code of the parsing method for
	the interconnect data file.
	From the file fragment in Figure \ref{subfig:ic_data} it can be seen that there
	are two parts in the interconnect data file.
	The first part contains the interconnect current data while the second part
	records the interconnect area data. Thus, after the extracted data is converted,
	the parsing method needs to check whether the data is related to the current or
	the area. Then the data is added to the corresponding list of the interconnect
	object.
	
	\setlength{\textfloatsep}{0.2cm}
\begin{algorithm2e}[H]
	\KwIn{an interconnect data file, an interconnect object}
	\KwOut{void}
	\While{not end of the file}
	{
		read one line\;
		\If{relevant data is contaied in the line}
		{
			extract the data\;
			convert the extracted data into the corresponding type\;
			\eIf{is current data}
			{
				add the converted data into the current list of the interconnect object\;
			}
			{
				add the converted data into the area list in of interconnect object\;
			}
		}
	}
	\caption{Parse Interconnect Data File}
	\label{algo:ic_parser}
\end{algorithm2e}
\setlength{\textfloatsep}{0.2cm}
	
	\section{Abstract SA design flow}
	\label{sec:sa_design_flow}
	The major design of the simulated annealing for the memory power optimization is
	the algorithm structure. Before the design details are introduced, an abstract
	design flow of the simulated annealing algorithm is provided in Figure
	\ref{fig:sa_framwork}. From the figure it can be seen that the algorithm is
	consisted of three parts.
	In the initialization of the algorithm, $S_{curr}$ is
	the current solution. It is set to an initial solution $S_{0}$. $C_{curr}$ is the
	cost of $S_{curr}$ and it is calculated according to $S_{0}$. The temperature
	$T$ is set to an initial value $T_{0}$.
	The second part of the algorithm is the inner loop. $S_{neigh}$ is the neighboring
	solution generated by the method $Neighbor()$ based on $S_{curr}$. Its cost
	$C_{neigh}$ is calculated by the cost function $Cost()$. Then the metropolis
	criterion $Metropolis()$ is performed to update $S_{curr}$ according
	to $C_{curr}$ and $C_{neigh}$.
	$Termination_{inner}()$ is the termination mechanism for the inner loop.
	The last part of the algorithm is the outer loop. $CoolingSchedule()$ is used to
	decrease $T$. $Termination_{outer}()$ is the termination mechanism for the outer loop.
	\begin{figure}[htb]
		\begin{center}
			\includegraphics[width=0.7\textwidth]{sa_flow}
			\caption{Simulate Annealing Design Flow}
			\label{fig:sa_framwork}
		\end{center}
	\end{figure}	

	The design of the simulated annealing should contains the answers of the following
	Questions \ref{ques:1} to \ref{ques:7}. To enhance the readability of the rest of this
	chapter, the design details will answer these questions sequentially.
	By providing a solution for each of these questions, the design for the
	simulated annealing algorithm is completed.
	\begin{enumerate} [\text{Question} 1:]
		\item What is the representation of solutions?
		\label{ques:1}
		\item How to generate neighboring solutions?
		\label{ques:2}
		\item What is the cost function?
		\label{ques:3}
		\item How to implement the metropolis criterion?
		\label{ques:4}
		\item What is the cooling schedule?
		\label{ques:5}
		\item How to determinate the initial solution and the initial temperature?
		\label{ques:6}
		\item How to terminate the inner loop and the outer loop?
		\label{ques:7}
	\end{enumerate}

	\section{Allocation and binding in single SA}
	\label{sec:stage_1}
	This approach is to seek for the optimal memory allocation and the
	binding of profiles simultaneously in a single simulated annealing
	algorithm.
	Section \ref{subsec:design_1} gives the answers of the design
	questions and Section \ref{subsec:problem_1} discusses the
	problems of this approach.
		\subsection{Design process}
		\label{subsec:design_1}
		Question \ref{ques:1}: What is the representation of solutions?
		
		As discussed in Section \ref{sec:memory_partition}, the result of the memory
		partitioning process is a configuration for the memory system.
		The solution provided by the simulated annealing algorithm should correspond
		to the memory configuration. According to this, solutions of the simulated
		annealing algorithm are definedas following.
		A solution is in the form of an integer vector that are divided
		into two parts. The first part of the vector corresponds to the allocation
		of memories.
		Each element in this part is the number of memory instances of the corresponding
		memory type.
		The second part of the vector contains the information about the binding of profiles.
		Each element in this part is the index of the memory type to which the corresponding
		profile is bound.
		Therefore, the vector length is the sum of the number of memory types and the
		number of profiles.
		To be clear, the first part of the vector is called the allocation part of
		the solution and the second part of the vector is called the binding part
		of the solution. The vector length is called the solution size.
		Figure \ref{fig:solu_exam_1} shows a solution example with two memory types and
		three application profiles.
		In the allocation part of the solution, one instance is allocated for both
		memory types.
		In the binding part, profile 0 is bound to memory type 0.
		Profile 1 and 2 are both bound to memory type 1.
		\begin{figure}[h]
			\begin{center}
				\includegraphics[width=0.7\textwidth]{solution_example_1}
				\caption{A Solution Example}
				\label{fig:solu_exam_1}
			\end{center}
		\end{figure}

		Question \ref{ques:2}: How to generate neighboring solutions?
		
		According to the solution representation, the neighboring solutions are described
		as following. 
		A neighboring solution is a vector with same length of the
		original solution. Only one element in the neighboring solution is different from the
		original solution.
		Figure \ref{fig:neigh_solu_exam_1} shows an example of neighboring solutions.
		The original solution is the same in Figure \ref{fig:solu_exam_1}.
		In neighboring solution 1, two memory instances are allocated for memory type 0
		while only one instance is allocated in the original solution.
		The rest elements of neighboring solution 1 are the same with the original solution.
		In neighboring solution 2, the only difference to the original solution is that
		profile 2 is bound to memory type 0.
		\begin{figure}[h]
			\begin{center}
				\includegraphics[width=0.7\textwidth]{neigh_example_1}
				\caption{Neighboring Solutions Example}
				\label{fig:neigh_solu_exam_1}
			\end{center}
		\end{figure}

		In this approach, the basic idea of $Neighbor()$ is to modify one element from the
		current solution. When generating neighboring solutions, the constraints of the formal
		power model should be taken into consideration.
		To ensure the constrain 1 and 3, the modification of the chosen element is limited in a
		range. As the solution is consisted of two different parts, the ranges of these
		two parts are also different.
		Let $Range_{\alpha}$ and $Range_{\beta}$ denote the modification ranges for
		the allocation part and the binding part respectively.
		The lower bound of $Range_{\alpha}$ is 0. The upper bound of $Range_{\alpha}$ is
		computed as following.
		Let $mems_{total}$ denote the total number of memory instances in the current solution.
		Suppose element $i$ in the allocation part is selected to be modified.
		Let $mem_{i}$ denotes the number of instances of memory type $i$ in the current solution.		
		Then the upper bound of $Range_{\alpha}$ equals $(mems_{max}-mems_{total}+mem_{i})$.
		$mems_{max}$ is the predefined constraint in the Equation \ref{equa:constraint_1}.
		For $Range_{\beta}$, its lower bound is 0.
		Let $Num_{memtype}$ denote the total number of memory types in the memory set.
		The upper bound of $Range_{\beta}$ is $(Num_{memtype}-1)$.
		
		Algorithm \ref{algo:neighbor_1} is the pseudo-code of $Neighbor()$.
		In the algorithm, there is another method $ConstraintsCheck()$.
		This method is used to examine whether the generated neighboring solutions satisfy the
		constrain 2 (Equation \ref{equa:constraint_2}) and constraint 4
		(Equation \ref{equa:constraint_4}) of the power model.
		Algorithm \ref{algo:constrain_check_1} shows the pseudo-code of $ConstraintsCheck()$.
		It first checks whether the area constraint $area_{max}$ is satisfied
		or not.
		$area_{total}$ is the total area consumed by the allocated memory instances and the interconnect.
		Then for each memory type, $ConstraintsCheck()$ examines whether enough memory space is
		provided for the profiles that are bound to the memory type.
		\setlength{\textfloatsep}{0.2cm}
\begin{algorithm2e}[t]
	\KwIn{$S_{curr}$}
	\KwOut{$S_{neigh}$}
	create a new solution $S_{neigh}$\;
	\While{ConstraintsCheck($S_{neigh}$) is false}
	{
		$S_{neigh}=S_{curr}$\;
		randomly selecte an element $element_{i}$ in $S_{neigh}$\;
		\eIf{$element_{i}$ is in the allocation part}
		{
			$element_{i}$ = Rondom number in $Range_{\alpha}$\;
		}
		{
			$element_{i}$ = Rondom number in $Range_{\beta}$\;
		}
	}
	\Return $S_{neigh}$ \;
	\caption{$Neighbor()$}
	\label{algo:neighbor_1}
\end{algorithm2e}
\setlength{\textfloatsep}{0.2cm}
		\setlength{\textfloatsep}{0.2cm}
\begin{algorithm2e}[h]
	\KwIn{$S_{neigh}$}
	\KwOut{a boolean value}
		compute $area_{total}$\;
		\eIf{$area_{total}>area_{max}$}
		{
			\Return false\;
		}
		{
			\ForEach {memory type $mt_{i}$}
				{
					find all profiles bound to $mt_{i}$\;
					compute the memory $mem_{required}$ required by these profiles\;
					compute the memory $mem_{provided}$ provided by $mt_{i}$\;
					\If{$mem_{provided}<mem_{required}$}
					{
						\Return false\;
					}
				}
				\Return true\;
		}
	\caption{$ConstraintsCheck()$}
	\label{algo:constrain_check_1}
\end{algorithm2e}
\setlength{\textfloatsep}{0.2cm}

		Question \ref{ques:3}: What is the cost function?
		
		Because the solution of the simulated annealing algorithm corresponds to the
		memory configuration, the solution cost $C$ is defined as the average power
		consumption of the configuration. The cost function $Cost()$ is designed
		based on the formal power model that is introduced in Section
		\ref{sec:memory_partition}. All the relevant parameter data required by the
		power model can be fetched from the input parameter organization.
	
		Question \ref{ques:4}: How to implement the metropolis criterion?
	
		The procedure of the metropolis criterion is straightforward and its
		implementation is illustrated in Algorithm \ref{algo:metropolis}.
		Because the goal of the power memory optimization is to reduce the
		power consumption, the solution with lower cost is considered as the
		better one in the metropolis criterion.
		A random real number $R$ is generated by the method $Random()$.
		This method only generates real numbers that are uniformly distributed in the
		range of 0 to 1 due to that $R$ is compared with the acceptance probability
		$P_{accept}$.
	
		\setlength{\textfloatsep}{0.2cm}
\begin{algorithm2e}[h]
	\KwIn{$C_{curr},C_{neigh},T$}
	\KwOut{void}
	\eIf{$C_{curr} \leq C_{neigh}$}
	{
		$S_{curr}=S_{neigh}$\;
		$C_{curr}=C_{neigh}$\;
	}
	{
		$P_{accept}=\exp{\left( -\frac{C_{neigh}-C_{curr}}{T} \right)}$\;
		$R = Random()$\;
		\If{$R \leq P_{accept}$}
		{
			$S_{curr}=S_{neigh}$\;
			$C_{curr}=C_{neigh}$\;	
		}
	}
	\caption{Metropolis Criterion Precedure}
	\label{algo:metropolis}
\end{algorithm2e}
\setlength{\textfloatsep}{0.2cm}
	
		Question \ref{ques:5}: What is the cooling schedule?
		
		For the cooling schedule, the temperature $T$ is linearly reduced with a fixed
		cooling ration $R_{cool}$.
		Equation \ref{equa:cooling_sche} is the design of $CoolingSchedule()$, where
		$N$ is the number of outer loop iterations.
		The value of the $R_{cool}$ should be a positive real number and
		it must be less than 1 in order to decrease $T$.
		However, it is a non-trivial task to determine a proper value for $R_{cool}$
		In this approach, the value of $R_{cool}$ is set to 0.9 first and it can be
		adjusted base on experiments.
		\begin{equation}
		\label{equa:cooling_sche}
			T_{N+1}=R_{cool} \cdot T_{N}
		\end{equation}
		
		Question \ref{ques:6}: How to determinate the initial solution and the
		initial temperature?
		
		The initial solution $S_{0}$ of the simulated annealing algorithm can be set
		to a given solution manually.
		In this approach, $S_{0}$ is set as following.
		Only one larger enough memory instance is allocated so that all application
		profiles can be bound to it.
		
		For the determination of the initial temperature $T_{0}$, the initial acceptance
		probability $P_{0}$ is defined.
		As Kirkpatrick et al. propose in the original article,
		$P_{0}$ is the expected acceptance probability of worser solutions at $T_{0}$
		\cite{10.2307/1690046}.
		In this approach, $T_{0}$ is computed by Equation \ref{equa:t_0_1} based
		on the metropolis criterion (Equation \ref{equa:metropolis_equation}).
		In the equation, $\left( C_{neigh}-C_{curr} \right)_{max}$
		is the maximum cost difference between a worser neighboring solution and the current
		solution.
		It can be measured by generating a set of worser neighboring solutions of the
		current solution randomly.
		The value of $P_{0}$ should be a positive real number less than 1.
		It should be close to 1 because the neighboring solutions are randomly
		accepted at $T_{0}$.
		$P_{0}$ is set to 0.9 in this approach and it can be adjusted based on experiments.
		\begin{equation}
		\label{equa:t_0_1}
			T_{0}= - \frac{\left( C_{neigh}-C_{curr} \right)_{max} }{ln{P_{0}}}
		\end{equation}
		
		Question \ref{ques:7}: How to terminate the inner loop and the outer loop?
		
		For the inner loop of the simulated annealing algorithm, the
		number of iterations is limited to a maximum number $Max_{iteration}$.
		During the execution of the inner loop, the number of iterations
		$Num_{iteration}$ is counted.
		When $Num_{iteration}$ becomes larger than $Max_{iteration}$, the inner
		loop terminates.
		The value of $Max_{iteration}$ is related to the size of the
		neighborhood $Size_{neighbor}$. $Size_{neighbor}$ is defined as the number
		of neighboring solutions of the current solution.
		In this approach, it is fixed to the solution size for simplification.
		Equation \ref{equa:iteration_innerloop} illustrates how to calculate the
		value of $Max_{iteration}$.
		From the equation it can be seen that $Max_{iteration}$ is proportional
		to $Size_{neighbor}$ with a factor $F_{iteration}$.
		The value of $F_{iteration}$ should be a positive real number.
		In this approach, $F_{iteration}$ is set to be 1 and it can also be
		modified based on experiments.
		\begin{equation}
		\label{equa:iteration_innerloop}
			Max_{iteration}=F_{iteration} \cdot Size_{neighbor}
		\end{equation}
		
		For the outer loop termination, a low temperature limit $T_{low}$ is defined.
		When the temperature $T$ is decreased to $T_{low}$, the outer loop terminates.
		The determination of $T_{low}$ is similar to the determination of $T_{0}$.
		$P_{low}$ is defined as the expected acceptance probability at $T_{low}$.
		Then $T_{low}$ is computed according to Equation \ref{equa:t_low_1}.
		In the equation, $\left( C_{neigh}-C_{curr} \right)_{min}$ is the minimum cost
		difference between a worser neighboring solution and the current solution.
		It can be measured by generating a set of worser neighboring solutions randomly
		as well. The value of $P_{low}$ should be a positive real number close to 0
		because the simulated annealing algorithm behaves like the local search algorithm
		at $T_{low}$. The worser neighboring solutions are seldom accepted.
		In this approach, the value of $P_{low}$ is set to 0.1 and it can be adjusted
		base on experiments.
		\begin{equation}
		\label{equa:t_low_1}
			T_{low}= - \frac{\left( C_{neigh}-C_{curr} \right)_{min}}{ln{P_{low}}}
		\end{equation}	
	
		\subsection{Problem discussion}
		\label{subsec:problem_1}
		For the evaluation of the design approach, the input data and the ILP results
		provided by the authors of \cite{Strobel2016} are used.
		The input data includes the parameters of 79 different memory types
		and the parameters of the interconnect.
		There are 4 applications, namely $IP reassembly$, $IP check$, $MD5$ and
		$Huffman$ \cite{Strobel2016}. Each application has $RAM$ and $ROM$ modes.
		The input data also contains the profile details for each of the applications.
		In this approach, the following testing scenario is designed.
		Parameter data of all 79 memory types along with the interconnect
		data are input to the simulated annealing algorithm.
		The application $IP reassembly$ in $ROM$ mode with its 241 profiles are used.
		The algorithm parameters are set according to the discussion in Section
		\ref{subsec:design_1}.
		The maximum number of instances $mems_{max}$ is limited to 8.
		There is no area constraint. $area_{max}$ is set to a large enough value.
		Figure \ref{fig:stage_1_result} shows the results of the designed simulated
		annealing algorithm and they are compared with the ILP results.
		After analyzing the algorithm execution process and its results,
		the following problems are found.
		\begin{figure}[h]
			\begin{center}
				\includegraphics[width=0.7\textwidth]{stage_1_result}
				\caption{Results Comparison Between Simulated Annealing and ILP}
				\label{fig:stage_1_result}
			\end{center}
		\end{figure}	
	
		Problem 1: Solutions always reaches $mems_{max}$.
		
		From Figure \ref{fig:stage_1_result} it can be seen that
		every solution provided by the designed simulated annealing
		algorithm contains 8 memory instances. This is due
		to the $ConstraintsCheck()$ in $Neighbor()$.
		When $S_{neigh}$ is generated by modifying an element in the allocation
		part of $S_{curr}$, $ConstraintsCheck()$ makes the probability of increasing
		the element higher than the probability of decreasing the element.
		Increasing the element satisfies the constraints because it allocates
		more memory instance than required by the profiles.
		However, decreasing the element may violate the constraints because it reduces
		the number of memory instances. This may result in that the memory type does
		not have enough memory space for the profiles.
		
		Another phenomenon found in
		the algorithm behavior is that the searching process will stick to a very small
		region once the solution allocates 8 memory instances. The reason of this
		phenomenon is described as following. Firstly, a memory type is called allocated
		if it has at least one memory instance. Let $memtypes_{allocated}$ denote the
		set of allocated memory types in $S_{curr}$. Let $memtypes$ denote the set of
		all memory types provided by the input data. According to the testing scenario,
		$\lvert memtypes_{allocated} \rvert \leq 8$, and $\lvert memtypes \rvert = 79$.
		Suppose $S_{curr}$ already contains 8 memory instances. $S_{neigh}$ is generated
		by modifying an element from the allocation part of $S_{curr}$.
		The probability of selecting one allocated memory type in $S_{curr}$ is
		$\dfrac{\lvert memtypes_{allocated} \rvert}{\lvert memtypes \rvert}$.
		In the best case, $\lvert memtypes_{allocated} \rvert = 8$.
		The probability is $\dfrac{8}{79} \approx 0.1$. This means that most of the time,
		$Neighbor()$ selects a memory type that is not allocated. It can neither increase
		nor decrease the element because there are already 8 memory instances allocated.
		Even if it is lucky to select an allocated memory type, the modification range
		$Range_{\alpha}$ of the element is rather small. For this reason, the searching
		process sticks to a small neighborhood of $S_{curr}$.
		
		Problem 2: Generating neighboring solutions is time consuming.
		
		After monitoring the execution time for different parts of the algorithm,
		it is found that the large fraction of the execution time is consumed
		by $Neighbor()$. It is still because of the $ConstraintsCheck()$ in $Neighbor()$.
		When generating $S_{neigh}$ by modifying an element in the binding part of
		$S_{curr}$, the probability of changing the selected element to the
		index of an allocated memory type is still calculated by
		$\dfrac{\lvert memtypes_{allocated} \rvert}{\lvert memtypes \rvert}$.
		In the testing scenario, it is $\dfrac{8}{79} \approx 0.1$ for the best case.
		This means that most of the time, the selected profile is bound to a memory
		type that is not allocated. Even if the profile is lucky to be bound to an
		allocated memory type, it may violate the constraints as well because the
		allocated memory type may have no enough space for it.
		Besides, Problem 1 also contributes to the time consumption.
		These two problems make the overall execution time of the designed simulated
		algorithm rather long.
		
		Problem 3: No convergence for the power cost.
		
		In Figure \ref{fig:stage_1_result}, the results of the simulated annealing
		is different for each execution.
		Because in each execution, the searching process sticks to different
		small region due to Problem 1. This is similar to the local optimum trap.
		Besides, the quality of the binding part is not guaranteed as well.
		
		The above problems make the designed simulated annealing algorithm an
		improper approach for the memory power optimization. However, in the next
		section, a nested simulated annealing structure is introduced to solve
		these problems.
		
	\section{Allocation and binding in nested SA}
	\label{sec:stage_2}
	From the approach discussed in Section \ref{sec:stage_1}, finding the optimal
	allocation and the binding at the same time seems to be problematical.
	In this approach, the memory partitioning is divided into two related sub-processes.
	One is the optimization for the memory allocation and the other one is the
	optimization for the profile binding.
	The allocation optimization is dependent on the result of the binding optimization.
	The simulated annealing algorithm is used for both sub-processes.
	To be simplified, let $SA_{\alpha}$ denote the simulated annealing algorithm for the
	allocation optimization and let $SA_{\beta}$ denote the simulated annealing algorithm
	for the binding optimization. Then the basic idea of this approach is to use $SA_{\beta}$
	as the cost function of $SA_{\alpha}$.
	Section \ref{subsec:design_2} answers the seven design questions.
	Section \ref{subsec:problem_2} discusses the problems of this approach.
	
		\subsection{Design process}
		\label{subsec:design_2}
		Question \ref{ques:1}: What is the representation of solutions?
		
		In the approach discussed in Section \ref{sec:stage_1}, the solution for the
		simulated annealing algorithm is consisted of the allocation part and the
		binding part. In this approach, the allocation part is defined as the
		solution of $SA_{\alpha}$ while the binding part is defined as the solution
		of $SA_{\beta}$. To be simplified, let $S_{\alpha}$ and $S_{\beta}$ denote
		the solutions of $SA_{\alpha}$ and $SA_{\beta}$ respectively.
		
		Question \ref{ques:2}: How to generate neighboring solutions?
		
		Let $Neighbor_{\alpha}$ and $Neighbor_{\beta}$ denote the neighboring
		solution generating method of $SA_{\alpha}$ and $SA_{\beta}$ respectively.
		Instead of modifying the chosen element according to a limited range,
		$Neighbor_{\alpha}$ increases or decreases the element by one. 
		The probabilities of the increase and decrease are both 0.5.
		But if the original value of the chosen element is 0, it can only be
		increased because the element should be positive.
		Algorithm \ref{algo:neighbor_alfa} is the pseudo-code for $Neighbor_{\alpha}()$.
		In the algorithm, $S_{\alpha,curr}$ is the current solution of
		$SA_{\alpha}$ and $S_{\alpha,neigh}$ is its neighboring solution.
		Similar to the Algorithm \ref{algo:neighbor_1}, the method
		$ConstraintsCheck_{\alpha}()$ is used to check whether $S_{\alpha,neigh}$
		satisfies the constraints or not.
		In this approach, the constraints of the power model are divided to two
		groups as well.
		Specially, a two-step implementation of the constraint 4 verification
		is made in this approach.
		Let $space_{provided}$ denotes the memory spaces provided by all allocated
		memory types in the current solution.
		Let $space_{required}$ denote the memory space required by all profiles.
		Then the first step of constraint 4 verification is to examine whether
		$space_{provided}$ is less than $space_{required}$ or not. This step is
		called constraint 4-1 verification.
		The second step is the same procedure of $ConstraintsCheck()$
		(Algorithm \ref{algo:constrain_check_1}) discussed in Section \ref{subsec:design_1}.
		But it excludes the checking for area constraint. This step is called
		constraint 4-2 verification.
		Then constraint 1, 2 and 4-1 are grouped together and they are called $Constraint_{\alpha}$.
		Constraint 3 and 4-2 are called $Constraint_{\beta}$.
		Algorithm \ref{algo:constrain_check_a} shows the pseudo-code for $ConstraintsCheck_{\alpha}()$.
		In the algorithm, $mems_{total}$ is the total number of memory instances allocated in
		$S_{\alpha,neigh}$ and $area_{total}$ is the total area consumed by the memory
		instances and the interconnect.
		$mems_max$ and $area_max$ are the predefined constraints in
		the formal power model.
		\setlength{\textfloatsep}{0.2cm}
\begin{algorithm2e}[htb]
	\KwIn{$S_{\alpha,curr}$}
	\KwOut{$S_{\alpha,neigh}$}
		create a new solution $S_{\alpha,neigh}$\;
		\While{$ConstraintsCheck_{\alpha}$($S_{\alpha,neigh}$) is false}
		{
			$S_{\alpha,neigh}=S_{\alpha,curr}$\;
			randomly selecte an element $element_{i}$ in $S_{\alpha,neigh}$\;
			increase or decrease $element_{i}$ by 1\;
		}
		\Return $S_{\alpha,neigh}$ \;
	\caption{$Neighbor_{\alpha}()$}
	\label{algo:neighbor_alfa}
\end{algorithm2e}
\setlength{\textfloatsep}{0.2cm}


		\setlength{\textfloatsep}{0.2cm}
\begin{algorithm2e}[h]
	\KwIn{$S_{\alpha,neigh}$}
	\KwOut{a boolean value}
		compute $mems_{total}$\;
		compute $area_{total}$\;
		\eIf{$mems_{total} \leq mems_{max} \text{ $\&$ } area_{total} \leq area_{max}$}
		{
			\Return true\;
		}
		{
			\Return false\;
		}
	\caption{$ConstraintsCheck_{\alpha}()$}
	\label{algo:constrain_check_a}
\end{algorithm2e}
\setlength{\textfloatsep}{0.2cm}
	
		For $SA_{\beta}$, $Neighbor_{\beta}()$ is more complicated than
		$Neighbor_{\alpha}()$. It is based on the solution of $SA_{\alpha}$.
		The basic idea of $Neighbor_{\beta}()$ is to randomly select one profile
		and bind it to another allocated memory type.
		To make the $Neighbor_{\beta}()$ clear, an example is given in
		Figure \ref{fig:sa_beta_neigh_example}.
		From the figure it can be seen that there are three
		memory types in $S_{\alpha}$ and there are three profiles in $S_{\beta}$.
		The allocated memory types are memory type 0 and 2 because they have at least one instances.
		In the current solution of $SA_{\beta}$, profile 0 is bound to
		memory type 2.
		When generating neighboring solution 1, profile 0 is chosen and it is changed to
		be bound to memory type 0.
		In neighboring solution 2, profile 2 is changed to be bound to memory type 2
		as well.
		The selected profiles can not be bound to memory type 1 because it is not allocated.
		\begin{figure}[h]
			\begin{center}
				\includegraphics[width=0.7\textwidth]{sa_beta_neigh_example}
				\caption{Neighboring Solutions Example for $SA_{\beta}$}
				\label{fig:sa_beta_neigh_example}
			\end{center}
		\end{figure}		
	
		Algorithm \ref{algo:neighbor_beta} shows the pseudo-code of
		$Neighbor_{\beta}()$.
		In the algorithm, $memtypes_{allocated}$ is the set of allocated
		memory types in $S_{\alpha}$.
		$S_{\beta,curr}$ and $S_{\beta,neigh}$ are the current and 
		neighboring solutions of $SA_{\beta}$.
		The method $ConstraintsCheck_{\beta}$ is also used to check whether
		$S_{\beta,neigh}$ satisfy the $Constraint_{\beta}$. It is same to
		Algorithm
		\ref{algo:constrain_check_1} without the area constraint verification.
		Through binding the chosen profile to an allocated memory type,
		many invalid neighboring solutions are avoided. This reduces the
		execution time consumed by $Neighbor_{\beta}()$.		
		\setlength{\textfloatsep}{0.2cm}
\begin{algorithm2e}[h]
	\KwIn{$S_{\alpha},S_{\beta,curr}$}
	\KwOut{$S_{\beta,neigh}$}
	create a new solution $S_{\beta,neigh}$\;
	\While{$ConstraintsCheck_{\beta}$($S_{\beta,neigh}$) is false}
	{
		$S_{\beta,neigh}=S_{\beta,curr}$\;
		randomly selecte an element $element_{i}$ in $S_{\beta,neigh}$\;
		identify the set of allocated memory types $memtypes_{allocated}$ in $S_{\alpha}$\;
		randomly selecte an memory type $memtype_{j}$ in $memtypes_{allocated}$\;
		$element_{i}$ = index of $memtype_{j}$ in $S_{\alpha}$\;
	}
	\Return $S_{\beta,neigh}$ \;
	\caption{$Neighbor_{\beta}()$}
	\label{algo:neighbor_beta}
\end{algorithm2e}
\setlength{\textfloatsep}{0.2cm}


	
		Question \ref{ques:3}: What is the cost function?
	
		In $SA_{\beta}$, the solution cost is still the memory power consumption
		that can be calculated according to the power model. The cost function is
		the same one introduced in the first approach.
		
		In $SA_{\alpha}$, there is no real cost for the its solution $S_{\alpha}$
		because $S_{\alpha}$ only contains the information about the memory allocation.
		Thus, the memory power consumption can not be calculated according to the
		power model.
		However, the cost function of $SA_{\alpha}$ evokes the procedure of $SA_{\beta}$.
		Then $SA_{\beta}$ optimizes the binding based on the current solution
		of $SA_{\alpha}$. After $SA_{\beta}$ terminates, it provides a binding with
		the minimum memory power consumption to the cost function of $SA_{\alpha}$.
		$SA_{\alpha}$ sets the minimum power consumption as its solution cost.
		Figure \ref{fig:nested_call} shows the nested simulated annealing procedure.
		In the figure, $C_{\alpha}$ and $C_{\beta}$ are the solution cost for $SA_{\alpha}$
		and $SA_{\beta}$ respectively.
		\begin{figure}[htb]
			\begin{center}
				\includegraphics[width=0.7\textwidth]{nested_call}
				\caption{Nested Simulated Annealing Procedure}
				\label{fig:nested_call}
			\end{center}
		\end{figure}

		Question \ref{ques:4}: How to implement the metropolis criterion?

		In this approach, the metropolis criterion of both $SA_{\alpha}$ and
		$SA_{\beta}$ are implemented the same with the first approach.
		
		Question \ref{ques:5}: What is the cooling schedule?
		
		The same cooling schedule (Equation \ref{equa:cooling_sche}) introduced
		in the first approach is used for both $SA_{\alpha}$ and $SA_{\beta}$.
		
		Question \ref{ques:6}: How to determinate the initial solution and the
		initial temperature?
		
		Both $SA_{\alpha}$ and $SA_{\beta}$ use the same mechanism introduced in
		the first approach.
		
		Question \ref{ques:7}: How to terminate the inner loop and the outer loop?
	
		For the inner loop termination, Equation \ref{equa:iteration_innerloop} is still used.
		However, the neighborhood sizes of $SA_{\alpha}$ and $SA_{\beta}$
		are described as following.
		The neighborhood size $Size_{neighbor,\alpha}$ is calculated according
		to Equation \ref{equa:neighbor_size_a}.
		The length of $S_{\alpha}$ is actually the total number of memory types
		in the memory set.
		\begin{equation}
		\label{equa:neighbor_size_a}
			Size_{neighbor,\alpha}=2 \cdot \text{length of } S_{\alpha} 
		\end{equation}
		Because $SA_{\beta}$ is embedded in $SA_{\alpha}$, $Size_{neighbor,\beta}$ is
		based on the solution of $SA_{\alpha}$.
		It is calculated according to Equation \ref{equa:neighbor_size_b}.
		In the equation, $\lvert memtypes_{allocated} \rvert$ is the number of
		allocated memory types in the current solution of $SA_{\alpha}$.
		The length of $S_{\beta}$ is actually the total number of profiles of the application.
		\begin{equation}
		\label{equa:neighbor_size_b}
			Size_{neighbor,\beta}=\lvert memtypes_{allocated} \rvert \cdot \text{length of } S_{\beta} 
		\end{equation}
		
		For the outer loop termination, $T_{low}$ is still used for both
		$SA_{\alpha}$ and $SA_{\beta}$.
		The determination of $T_{low}$ is the same discussed in the first approach.
	
		\subsection{Problem discussion}
		\label{subsec:problem_2}
		The implementation of the nest simulated annealing algorithm is divided to two steps.
		The fist step is to implement $SA_{\beta}$ with a given valid memory allocation.
		For the evaluation of $SA_{\beta}$, the same testing scenario introduced in approach 1
		is used. The best memory allocation provided by the ILP approach is given to $SA_{\beta}$.
		The results of $SA_{\beta}$ are shown in
		Figure \ref{fig:stage_2_results} and they are compared with the ILP results.
		From the figure, it can be seen that $SA_{\beta}$ can yield the optimal or the
		near optimal of solutions under the precondition that the optimal solution of
		$SA_{\alpha}$ is given.
		However, the following problems are found during the implementation of $SA_{\alpha}$.
		\begin{figure}[htb]
			\begin{center}
				\includegraphics[width=0.7\textwidth]{stage_2_results}
				\caption{Results Comparison Between $SA_{\beta}$ and ILP}
				\label{fig:stage_2_results}
			\end{center}
		\end{figure}		
		
		Problem 1: The execution time of the nest simulated annealing is unacceptable in
		practice.
		
		Because $SA_{\alpha}$ and $SA_{\beta}$ is nested, the overall execution time
		$t_{overall}$ can be calculated according to Equation \ref{equa:nested_time}.
		In the equation, $iteration_{out,\alpha}$ and $iteration_{in,\alpha}$ are the
		number of the outer loop iterations and the number of the inner loop iterations in $SA_{\alpha}$.
		$t_{\beta}$ is the execution time of $SA_{\beta}$.
		According to Equation \ref{equa:iteration_innerloop} and Equation
		\ref{equa:neighbor_size_a},
		$iteration_{in,\alpha} $ equals to $2 \cdot \text{ length of $S_{\alpha}$ } \cdot F_{iteration}$.
		Based on the testing scenario, $\text{length of $S_{\alpha}$}=79$ and $F_{iteration}=1$.
		Thus, $iteration_{in,\alpha}$ equals to 158.
		The measured average $t_{\beta}$ is around 30 seconds.
		Assume $iteration_{out,\alpha}=1$, then
		$t_{overall}=1 \cdot 158 \cdot 30=4740$ seconds $\approx 1.32$ hours.
		This is the overall execution time when the outer loop of $SA_{\alpha}$ is
		performed only once. It is already an unexpected time for the optimization
		process.
		However, the average $iteration_{out,\alpha}$ in the testing scenario is always
		greater than 50.
		Even if $t_{\beta}$ is decreased by optimizing $SA_{\beta}$ from programming
		side, $t_{overall}$ is still unacceptable in practice.
		The cost function in the simulated annealing algorithm should be the mathematical
		computations that can be executed very fast by computers.
		However, the cost function in $SA_{\alpha}$ is a simulated annealing algorithm
		in this approach. This is the reason of this problem.
		\begin{equation}
		\label{equa:nested_time}
			t_{overall}=iteration_{out,\alpha} \cdot iteration_{in,\alpha} \cdot t_{\beta}
		\end{equation}
		
		Problem 2: Invalid $S_{\alpha}$ is provided to $SA_{\beta}$.
		
		The verification of constraint 4-1 can not guarantee the validity of $S_{\alpha}$.
		In $ConstraintsCheck_{\alpha}$ (Algorithm \ref{algo:constrain_check_a}),
		the verification of constraint 4-1 only compares $space_{provided}$ and
		$space_{required}$. However, satisfying constraint 4-1 does not ensure $S_{\alpha}$
		is valid. There is one fact ignored by constraint 4-1. The fact is that one profile
		can not be distributed to different memory types. Consider the following situation.
		There are two memory types provided, namely $M_{512}$ and $M_{256}$. The subscripts
		of their names are the memory sizes. Suppose a profile $P$ with size 700 needs to
		be bound to the memory type and only 2 instances are allowed.
		Figure \ref{fig:cons_4_1_invalid} shows the different combinations of the memory
		instances and their validities according to constraint 4-1 and the original
		constraint 4. In the case that one instance is allocated for both memory types,
		the allocation is valid according to constraint 4-1 because
		$space_{provided} > space_{required}$. However, this allocation is not valid
		according to the original constraint 4 due to the fact that profiles can not
		be distributed to multiple memory types.
		\begin{figure}[htb]
			\begin{center}
				\includegraphics[width=0.7\textwidth]{constraint_4_1}
				\caption{Example For Generating Invalid $S_{\alpha}$}
				\label{fig:cons_4_1_invalid}
			\end{center}
		\end{figure}
	
		Because of the above problems, this approach only obtains the success for
		$SA_{\beta}$. Due to the unacceptable execution time, the
		evaluation of $SA_{\alpha}$ is not performed.
		In the next section, an enhancement of the nested simulated
		annealing algorithm is introduced.
		
	\section{Allocation and binding in independent SAs}
	\label{sec:stage_3}
	This approach is to solve the problems of nested simulated annealing algorithm
	discussed in Section \ref{sec:stage_2}. The basic idea of this approach is
	still using $SA_{\alpha}$ and $SA_{\beta}$ to optimize the memory allocation
	and the profile binding respectively. However, $SA_{\alpha}$ and $SA_{\beta}$
	are independent from each other.
	
	Instead of evoking $SA_{\beta}$, $SA_{\alpha}$ in this approach calculates
	the memory power consumption in its cost function.
	But the power consumption can not be calculated with only the memory
	allocation.
	To solve this problem, the method $SortedBind()$ is used to
	provide an assumed binding information to $SA_{\alpha}$.
	Another purpose of $SortedBind()$ is to avoid generating invalid $S_{\alpha}$
	due to constraint 4-1 verification.
	Algorithm \ref{algo:sort_bind} shows the pseudo-code of $SortedBind()$.
	In the algorithm, $d$ is the profile duty cycle.
	$p_{r}$ and $p_{w}$ are the profile read and write probability respectively.
	The free space of a memory type is the memory space that is not be occupied.
	The details of $SortedBind()$ and its basis are introduced later in Section
	\ref{subsec:design_3}.
	\setlength{\textfloatsep}{0.2cm}
\begin{algorithm2e}[htb]
	\KwIn{$S_{\alpha}$,$S_{\beta}$}
	\KwOut{a boolean value}
		sort memory types in ascending order according to size\;
		\eIf{ROM mode}
		{
			sort profiles in descending order according to $d \cdot p_{r}$\;
		}
		{
			sort profiles in descending order according to $d \cdot \left( p_{r} + p_{w} \right)$\;
		}
		\ForEach {$profile_{i}$ in $S_{\beta}$}
		{
			in $S_{\alpha}$, find the smallest memory type $mt$ that has enough free space for $profile_{i}$\;
			bind $profile_{i}$ to $mt$\;
		}
		\eIf{$S_{\beta}$ satisfies constraint 4}
		{
			\Return true\;
		}
		{
			\Return false\;
		}
	\caption{$SortedBind()$}
	\label{algo:sort_bind}
\end{algorithm2e}
\setlength{\textfloatsep}{0.2cm}

	
	The common problem of the first two approaches
	is that the number of the allocated memory instances always reaches constraint
	$mems_{max}$.
	To avoid this problem, this approach introduces the level concept to the memory
	allocation optimization. In each allocation optimization level, only a fixed
	number of memory instances are allowed. Let $mems_{allowed}$ denote this fixed
	number. Then only the allocations with exact $mems_{allowed}$ instances are
	taken into consideration for a certain level.
	For example, suppose $mems_{allowed}=4$ in an allocation optimization level.
	Then all the potential allocations contain exactly 4 memory instances.
	$SA_{\alpha}$ is performed for each optimization level to obtain the corresponding
	optimal memory allocation. Then $SA_{\beta}$ is used to seek for the optimal
	profile binding of these allocations. Lastly, the pair of allocation and binding
	with minimum power cost is output as the final result of the optimization process.
	Figure \ref{fig:sa_level} shows an allocation optimal level structure with the
	constraint $mems_{max}=4$.
	
	This approach mainly focus on the design of $SA_{\alpha}$ and the enhancement
	for $SA_{\beta}$ because $SA_{\beta}$ can already yield good results.
	In the following, the seven design questions are answered in Section
	\ref{subsec:design_3}. Section \ref{subsec:problem_3} discusses the findings
	and problems of this approach.
	\begin{figure}[h]
		\begin{center}
			\includegraphics[width=0.7\textwidth]{sa_level}
			\caption{Allocation Optimization Levels with $mems_{max}=4$}
			\label{fig:sa_level}
		\end{center}
	\end{figure}

		\subsection{Design process}
		\label{subsec:design_3}
		Question \ref{ques:1}: What is the representation of solutions?
		
		In this approach, the representation of solutions for both $SA_{\alpha}$
		and $SA_{\beta}$ are the same with the nested simulated annealing algorithm.
		
		Question \ref{ques:2}: How to generate the neighboring solutions?
		
		Because the number of allocated memory instances is fixed for solutions in
		$SA_{\alpha}$, the method of generating $S_{\alpha}$ should follow this rule.
		There are two steps in $Neighbor_{\alpha}()$.
		The first step is to select one allocated memory type in $S_{\alpha,curr}$
		and decrease its number of instances by one.
		The second step is to choose an arbitrary memory type in $S_{\alpha,curr}$
		and increase its number of instances by one.
		Algorithm \ref{algo:sa_alfa_neigh_1} shows the pseudo-code of
		$Neighbor_{\alpha}()$.
		In the algorithm, $ConstraintsCheck_{\alpha}()$ is the same one in the nested
		simulated annealing algorithm. $SortedBind()$ is added to the check condition
		to guarantee the validity of $SA_{\alpha,neigh}$.
		\setlength{\textfloatsep}{0.2cm}
\begin{algorithm2e}[h]
	\KwIn{$S_{\alpha,curr}$, $S_{\beta}$}
	\KwOut{$S_{\alpha,neigh}$}
		create a $S_{\alpha,neigh}$\;
		\While{($ConstraintsCheck_{alpha}(S_{\alpha,neigh})$ $\&$ SortedBind($S_{\alpha,neigh}$,$S_{\beta}$)) is false}
		{
			$S_{\alpha,neigh}=S_{\alpha,curr}$\;
			select an allocated memory type $mt_{allocated}$ in $S_{\alpha,curr}$\;
			select an arbitrary memory type $mt_{arbitrary}$ in $S_{\alpha,curr}$\;
			delete an instance for $mt_{allocated}$ in $S_{\alpha,neigh}$\;
			add an instance for $mt_{arbitrary}$ in $S_{\alpha,neigh}$\;
		}
		\Return $S_{\alpha,neigh}$\;
	\caption{$Neighbor_{\alpha}$()}
	\label{algo:sa_alfa_neigh_1}
\end{algorithm2e}
\setlength{\textfloatsep}{0.2cm}
		
		For $SA_{\beta}$, neighboring solutions are generated by the same mechanism used
		in the nested simulated annealing (Algorithm \ref{algo:neighbor_beta}).
				
		Question \ref{ques:3}: What is the cost function?
		
		In this approach, the costs of both $S_{\alpha}$ and $S_{\beta}$
		are the memory power consumption. They can be calculated based on the formal
		power model. The cost function of $SA_{\beta}$ is the same one in the nested
		simulated annealing algorithm.
		However, the cost function of $SA_{\alpha}$ needs the binding information
		provided by $SortedBind()$ in order to calculate the memory power cost.
		
		The following discussion is about the basis of $SortedBind()$.
		In order to evaluate the memory allocation provided by $SA_{\alpha}$,
		the corresponding binding is required. But for a given allocation,
		there are lots of potential bindings of a set of application profiles.
		Different bindings yield different memory power costs. This will affect
		the evaluation results.
		Thus, a standard should be defined for selecting the bindings for the
		memory allocation. For the same set of application profiles, if the
		best profile binding is selected for all potential memory allocations, the
		evaluation becomes uniform. For defining the best binding, the following
		assumptions are made.
		The best application profile binding is obtained by mapping frequently
		accessed profiles to small memory types. The profile access frequency
		is dependent on the product of the profile duty cycle and the profile access
		probability. The profile access probability in ROM mode is the profile
		read probability. The profile access probability in RAM mode is the
		sum of profile read probability and write probability.
		These assumptions are made based on the principle of memory partitioning and the
		ILP results provided by the authors of \cite{Strobel2016}.
		In algorithm \ref{algo:sort_bind}, $SortedBind()$ sorts the memory types and
		the application profiles. Then it binds each profile to the memory types
		according to these assumptions. If all profiles are bound successfully,
		the quality of the memory allocation provided in $SA_{\alpha}$ can be
		evaluated based on this binding.
		 
		Question \ref{ques:4}: How to implement the metropolis criterion?
		
		The implementation of the metropolis criterion for both $SA_{\alpha}$
		and $SA_{\beta}$ are the same with the nested simulated annealing algorithm.
		
		Question \ref{ques:5}: What is the cooling schedule?
		
		The same cooling schedule in the nested simulated annealing algorithm is
		used for both $SA_{\alpha}$ and $SA_{\beta}$.
		
		Question \ref{ques:6}: How to determinate the initial solution and the
		initial temperature?
		
		In this approach, the initial solution for both $SA_{\alpha}$
		and $SA_{\beta}$ are generated randomly instead of being set manually.
		This is for the verification of the feature that simulated annealing
		is not sensitive to the initial solution.
		
		To improve the accuracy of initial temperature $T_{0}$, a second step is
		add to the determination of $T_{0}$. After computing $T_{0}$ according
		Equation \ref{equa:t_0_1}, the inner loop of the simulated annealing
		algorithm is performed once at $T_{0}$.
		During the execution of the inner loop, the acceptance probability is measured.
		Then the measured acceptance probability $P_{measured}$ is compared with the
		expected acceptance probability $P_{0}$. If $P_{measured}$ is smaller than
		$P_{0}$, $T_{0}$ is doubled. The same process is repeated until $P_{measured}$
		is larger than $P_{0}$. It makes sure that $T_{0}$ is high enough so that
		random solutions can be accepted in the metropolis criterion.
		
		Question \ref{ques:7}: How to terminate the inner loop and the outer loop?
		
		The inner loop termination mechanism of the nested simulated annealing algorithm
		is used for both $SA_{\alpha}$ and $SA_{\beta}$ in this approach.
		However, $Size_{neigh,\alpha}$ is different because $Neighbor_{\alpha}()$
		is modified in this approach. Equation \ref{equa:neigh_size_alfa_2}
		shows the computation for $Size_{neigh,\alpha}$ in this approach.
		$\lvert memtypes_{allocated} \rvert$ is the number of allocated memory types in
		the current solution of $SA_{\alpha}$. The length of $S_{\alpha}$ is actually
		the total number of memory types in the memory set. The computation of $Size_{neigh,\beta}$ is
		still the same one of the nested simulated annealing because $Neighbor_{\beta}()$
		is not changed in this approach.
		\begin{equation}
		\label{equa:neigh_size_alfa_2}
			Size_{neigh,\alpha}=\lvert memtypes_{allocated} \rvert \cdot \text{ length of } S_{\alpha}
		\end{equation}
		
		For the termination of the outer loop of simulated annealing algorithm,
		a new mechanism is introduced in this approach. During the execution of inner loop,
		the acceptance probability is measured. A low threshold $P_{threshold}$ for the
		acceptance probability is defined. If the measured acceptance probability is
		smaller than $P_{threshold}$ in five successive iterations, the outer loop
		terminates. The value of $P_{threshold}$ should be a small positive real number
		close to 0. In this approach, $P_{threshold}$ is set to $10^{-3}$ and it
		can be adjusted by experiments.
		
		\subsection{Problem discussion}
		\label{subsec:problem_3}
		For the evaluation of this approach, the same testing scenario introduced in the previous
		approach is used. As discussed in the nested simulated annealing algorithm,
		$SA_{\beta}$ can yield good enough profile bindings for a given memory allocation.
		Therefore, the functionality of $SA_{\alpha}$ is mainly focused in this approach.
		The evaluation details of $SA_{\beta}$ are represented in Chapter \ref{chap:evaluation}.
		
		To verify its functionality, $SA_{\alpha}$ is executed multiple times under the same
		testing scenario. Figure \ref{fig:stage_3_results_rom_ipres} shows the result.
		\begin{figure}[htb]
			\begin{center}
				\includegraphics[width=0.9\textwidth]{stage_3_results_rom_ipres}
				\caption{Memory Allocation Comparison Between $SA_{\alpha}$ and ILP (ROM)}
				\label{fig:stage_3_results_rom_ipres}
			\end{center}
		\end{figure}		
		From the figure it can be seen that the memory allocation provided by $SA_{\alpha}$ is
		same with the allocation provided by ILP. However, $SA_{\alpha}$ is executed with only
		one testing scenario. To verify its functionality further, the same application in
		RAM mode is used.
		Figure \ref{fig:stage_3_results_ram_ipres} shows the memory allocations provided by $SA_{\alpha}$
		and ILP. From the figure it is can be seen that the two memory allocations are different
		from each other. By analyzing the algorithm execution process and its results,
		the following problems are found.
		\begin{figure}[htb]
			\begin{center}
				\includegraphics[width=0.9\textwidth]{stage_3_results_ram_ipres}
				\caption{Memory Allocation Comparison Between $SA_{\alpha}$ and ILP (RAM)}
				\label{fig:stage_3_results_ram_ipres}
			\end{center}
		\end{figure}

		Problem 1: Lager memory types are always allocated.
		
		After executing $SA_{\alpha}$ multiple times, it is found that $SA_{\alpha}$ always
		allocates instances for memory types with larger size. For example, in Figure
		\ref{fig:stage_3_results_ram_ipres}, two instances are allocated for memory type with size
		4194304 in $SA_{\alpha}$, which is much larger than the memory type allocated in
		ILP. To illustrate this problem clearly, let $allocation_{SA}$ and $allocation_{ILP}$
		denote the memory allocations provided by $SA_{\alpha}$ and ILP respective. The data
		in Figure \ref{fig:stage_3_results_ram_ipres} is used for $allocation_{SA}$ and $allocation_{ILP}$.
		Assume the current solution of $SA_{\alpha}$ is $allocation_{SA}$ and the searching
		process is guided towards to $allocation_{ILP}$. In order to reach $allocation_{ILP}$,
		the following steps should be performed according to $Neighbor_{\alpha}$.
		The first step is to delete one instance from the four allocated memory types in $allocation_{SA}$.
		The second step is to add one instance for memory type MEM\_1048576\_16.
		These two steps are repeated until eight instances are allocated for MEM\_1048576\_16.
		In the testing scenario, the total number of memory types is 79.
		The probability of selecting MEM\_1048576\_16 out of 79 memory types is $\frac{1}{79}$.
		In addition, MEM\_1048576\_16 should be selected eight times continuously. The
		probability of moving from $allocation_{SA}$ to $allocation_{ILP}$ is
		$\frac{1}{79}^{8} \approx 0$. This means that it is almost impossible to reach
		$allocation_{ILP}$ from $allocation_{SA}$. This problem is caused by the mechanism
		for generating neighboring solutions in $SA_{\alpha}$.
		
		After examining the input application profile file for RAM carefully, another phenomenon is found.
		There are some profiles with extremely large size. For example, there is a profile with size 7973948
		in above testing scenario. In order to satisfy the constraints, at least one memory type with
		large enough memory space should be allocated. In Figure \ref{fig:stage_3_results_ram_ipres}, both
		$SA_{\alpha}$ and ILP provide such memory type. Suppose $Neighbor_{\alpha}()$ selects
		MEM\_4194304\_16 and deletes one instance first. If an instance is added for a memory type with
		size smaller than 7973948, the generated neighboring solution becomes invalid. Because
		there is no memory type providing enough memory space required by the large profile.
		Thus, $Neighbor_{\alpha}()$ always adds instances for lager memory types.
		
		Problem 2: Memories with same size but different sub-bankings can not be differentiated.
		
		In the method $SortedBind()$, the memory types are sorted according to their sizes.
		However, their sub-banking parameters are not taken into consideration. This will affect
		the solution quality provided by $SA_{\alpha}$. Figure \ref{stage_3_results_huffman_rom}
		shows the memory allocations provided by $SA_{\alpha}$ and ILP for $Huffman$
		application in ROM mode. From the figure, it is can be seen that $SA_{\alpha}$ allocates
		instances for MEM\_512\_2 while ILP allocates instances for MEM\_512\_1.
		Both memory types have the same size but their numbers of sub-bankings are different.
		\begin{figure}[htb]
			\begin{center}
				\includegraphics[width=0.9\textwidth]{stage_3_results_huffman_rom}
				\caption{Memory Allocation Comparison Between $SA_{\alpha}$ and ILP (Huffman, RAM)}
				\label{fig:stage_3_results_huffman_rom}
			\end{center}
		\end{figure}		
		
		Problem 3: This approach depends on unproven assumptions.
		
		The assumptions made for $SortedBind()$ in this approach are based on the principle and
		the observation of ILP results provided by the authors of \cite{Strobel2016}.
		Their usage should be verified scientifically.