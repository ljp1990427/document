\label{chap:introduction}
In the field of embedded systems design nowadays, power consumption
becomes one of the most important design factors especially in the domain
of Systems-on-Chip. One of the key issues to design power-efficient embedded
system is the power consumed by memories and memory related components.
Some researchers have claimed that large fraction of power is dissipated by
memories rr rr. Thus, memory power optimization plays a significant role
in the design of power-efficient embedded systems. One of the most effective
and common approaches to reduce memory power consumption is the memory 
partitioning method which is proposed in several articles rr,rr \cite{Strobel2016}.

The rationale of memory partitioning is to split one single large memory into 
several small memory instances which can be accessed individually. And according 
to the profiled memory access patterns, frequently accessed address ranges are 
grouped to smaller memory instances while seldom accessed address ranges are grouped 
to larger ones. Therefor, the memory power optimization can be achieved by the facts 
that smaller memory instances consume less power and the larger memory instances are 
seldom accessed.

%---------------------------------------------------------------------
\section{Motivation}
\label{sec:motivation}

%---------------------------------------------------------------------
% ...
