\label{chap:introduction}
	\section{Memory power optimization}
	\label{sec:memory_power_optim}
	In the field of embedded systems design nowadays, power consumption
	becomes one of the most important design factors especially in the domain
	of Systems-on-Chip. One of the important issues to design power-efficient embedded
	system is the power consumed by memories and memory related components.
	Some researchers have claimed that large fraction of power is dissipated by
	memories \cite{Strobel2016, 876761, 4415607}. Thus, memory power optimization plays 
	a significant role in the design of power-efficient embedded systems. One of the most 
	effective and common approaches to reduce memory power consumption is the memory 
	partitioning method which is proposed in several articles and books
	\cite[p.43]{Strobel2016, 876761, 4415607, Hiser:2005:EAP:1088093, macii2002memory}.

	The rationale of memory partitioning is , on the one hand, to split one single large memory into 
	several small memory instances which can be accessed individually. On the other hand, according 
	to the profiled memory access patterns, frequently accessed address ranges are 
	grouped to smaller memory instances while seldom accessed address ranges are grouped 
	to the larger ones. Therefor, the memory power optimization can be achieved by the facts 
	that smaller memory instances consume less power and the larger memory instances are 
	seldom accessed.
	There are two central concepts for memory power optimization using the memory partitioning method.
	One concept is the allocation $ \alpha $ which is a set of memory instances of certain memory types. Memory
	types are defined by the physical characteristic parameters such instance size, area, read current
	and so on.The other concept is the binding $ \beta $ of application code and data fragments to the 
	selected memory instances. The code and data fragments of an application are refereed as profiles
	of this application. And each application is represented by a set of profiles. Every profile is
	characterized by some user-defined parameters. Because all the code and data should be stored in
	the memories, each profile should be bound to exactly one memory instance \cite{Strobel2016}. 
	A configuration for the memory system is defined as the combination of an allocation of memory 
	instances and the corresponding binding for the application profiles. The goal of memory power 
	optimization is to find a configuration among all possible configurations such that the overall 
	power consumed by all selected memory instances is the lowest under certain predefined constraints.
	
	Obviously, the memory power optimization is one of the combinatorial optimizations since the process
	is to find the optimal solution from a finite solution space of a problem set. There are many algorithms
	that can be applied to solve this kind of problems in the domain of combinatorial optimization.
	One approach is the integer linear programming (ILP) which solves the optimization problem through a 
	mathematical model described by certain integer linear relationship. This approach is proposed in
	\cite{Strobel2016} to solve the memory power optimization problem. Another commonly applied approach is the
	heuristic algorithm which is based on searching mechanisms. Many classical combinatorial optimization
	problems such as traveling sales man (TSP) problem have been solved by using heuristic algorithms.
	
	Heuristic is a technique that searches for a near optimal solution of a optimization problem within
	a reasonable time. It is often used when the exact optimal solution can not be found by the conventional
	algorithms. When solving a optimization problem with a very large solution space, the algorithms trying to
	find the exact optimal solution may be ideal. However, the computation time of such algorithms may be not
	acceptable in practice. In such cases, the user of heuristic can find a good solution in a reasonable
	time. Though the solution provided by heuristic may not be the exact optimal one, it still can be
	considered as a valuable solution of the optimization problem. One key feature of heuristic is the
	trade-off between efficiency and precision. The solution quality and the computation time can be balanced by
	users according to their requirements.
	
	The aim of this thesis work is trying to select one promising heuristic for the purpose of solving the memory power optimization problem. Though the heuristics have been applied to solve a variety of optimization problems, a partial success is achieved in the
	memory power optimization due to the twoflod feature of the problem set. The rest of this work is organized as follows.
	Chapter 2 discusses the related researches for the memory power optimization and heuristics. In Chpater 3, some potential
	heuristics are theoretical examined and compared. Simulated annealing, the most promising algorithm, is proposed for the 
	optimization objective. A detailed introduction for the simulated annealing algorithm is given in Chpater 4. Chpter 5 represents
	the implementation process for optimizing memory power consumption using simulated annealing. Chapter 6 discusses the evaluation
	of the partial results and the conclusion is represented in Chapter 7.
%---------------------------------------------------------------------


%---------------------------------------------------------------------
% ...
