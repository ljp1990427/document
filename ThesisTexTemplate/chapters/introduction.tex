\label{chap:introduction}
	\section{Memory power optimization}
	\label{sec:memory_power_optim}
	In the field of embedded systems design nowadays, power consumption has become one of the most important design factors especially in the domain of Systems-on-Chip. One of the important issues to design power-efficient embedded system is the power consumed by memories and memory related components. Some researchers have claimed that large fraction of power is dissipated by
	memories \cite{876761, 4415607}. Thus, memory power optimization plays a significant role in the design of power-efficient embedded systems.One of the most effective and common approaches to reduce memory power consumption is the memory partitioning method which is proposed in several articles and books \cite[p.43]{Strobel2016, 876761, 4415607, Hiser:2005:EAP:1088093, macii2002memory}.
	The rationale of memory partitioning is , on the one hand, to split one single large memory into several small memory instances which can be accessed individually \cite{4415607}. On the other hand, according to the memory access patterns, frequently accessed address ranges are grouped to smaller memory instances while seldom accessed address ranges are grouped 
	to the larger ones \cite{Strobel2016}.
	The memory power optimization is one of the combinatorial optimizations since the process is to find the optimal solution from a finite solution space of a problem set. There are many algorithms that can be applied to solve this kind of problems in the domain of combinatorial optimization.
	One approach is the integer linear programming (ILP) which solves the optimization problem through a 
	mathematical model described by certain integer linear relationship. This approach is proposed in
	\cite{Strobel2016} for the memory power optimization. Another commonly applied approach is the
	heuristic algorithm which is based on searching mechanisms. Many classical combinatorial optimization
	problems such as traveling sales man (TSP) problem have been solved by using heuristic algorithms.
	
	Heuristic is a technique that searches for a near optimal solution of a optimization problem within
	a reasonable time. It is often used when the exact optimal solution can not be found by the conventional
	algorithms. When solving a optimization problem with a very large solution space, the algorithms trying to
	find the exact optimal solution may be ideal. However, the computation time of such algorithms may be not
	acceptable in practice. In such cases, the user of heuristic can find a good solution in a reasonable
	time. Though the solution provided by heuristic may not be the exact optimal one, it still can be
	considered as a valuable solution of the optimization problem. One key feature of heuristic is the
	trade-off between efficiency and precision. The solution quality and the computation time can be balanced by
	users according to their requirements.
	
	The goal of this thesis work is to apply heuristics for the purpose of memory power optimization. Firstly, multiple potential
	heuristics are theoretical examined and compared. Secondly, the most promising algorithm is identified and proposed for the
	optimization goal. Then the selected heuristic is adapted to a formal power model which is defined in \cite{Strobel2016}.
	Lastly,  
	
	
	
	
	The rest of this work is organized as follows.
	Chapter 2 discusses the related researches for the memory power optimization and heuristics. In Chpater 3, some potential
	heuristics are theoretical examined and compared. Simulated annealing, the most promising algorithm, is proposed for the 
	optimization objective. A detailed introduction for the simulated annealing algorithm is given in Chpater 4. Chpter 5 represents
	the implementation process for optimizing memory power consumption using simulated annealing. Chapter 6 discusses the evaluation
	of the partial results and the conclusion is represented in Chapter 7.
%---------------------------------------------------------------------


%---------------------------------------------------------------------
% ...
